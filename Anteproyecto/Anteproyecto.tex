\documentclass[12pt,a4paper]{report}
\begin{document}
\title{Sistema de adaptación inteligente de la velocidad para vehículos basados en inteligencia artificial y visión por computador}
\author{Sergio Sastre Arrojo}
\maketitle
\chapter*{Introducción}
\paragraph{En los últimos años, con la aparición de los sistemas ISA para conseguir que los vehículos adapten su velocidad se han evitado muchas desgracias afortunadamente. No obstante, han acarreado una serie de inconvenientes que se podrían mejorar. Por ejemplo: estos sistemas utilizan GPS, lo cual es muy eficiente para su propósito, pero en determinadas ocasiones (núcleos urbanos, distinción de carriles en una autovía con una carretera de servicio) pueden desembocar en situaciones de gran peligro.}
\paragraph{Es por ello que aquí presentamos una solución ante esos problemas: \[ISA^{2}\]}
\paragraph{Con este nuevo sistema queremos añadirle un factor más de reconocimiento para adaptar su velocidad, y no es otro que la situación del tráfico en cada momento para poder estimar una velocidad apropiada del vehículo.}
\paragraph{La primera versión ya se realizó en su momento, por lo que el objetivo de este proyecto es mejorarlo a partir de nuevas tecnologías que han ido surgiendo en los últimos años.}
\chapter*{Metodología}
\end{document}
