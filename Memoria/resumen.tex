\chapter{Resumen}
%\textcolor{red}{Resumen de 100 palabras MÁXIMO}\\

%En el siguiente trabajo vamos a tratar de solucionar un problema que lleva existiendo desde la invención del automóvil de una forma que no se había visto antes. Con la creación de las normas que regulan el tráfico en las carreteras y la fabricación masiva de vehículos, es un hecho que la probabilidad de que ocurra un accidente es muy alta. Para solucionarlo se inventaron los sistemas \ac{ISA} \cite{reduccion} que ayudaron a reducir este número... Sin embargo, no son perfectos y tienen fallos.

%Es por ello que aquí sugerimos nuestra solución con un nuevo y mejorado sistema: \[ISA^{2}\]
En el siguiente trabajo mostramos una versión mejorada del sistema de adaptación de la velocidad $ISA^{2}$. Éste consiste en la estimación apropiada de la velocidad para un vehículo basándose en la situación del tráfico en un determinado momento capturada por una cámara. A diferencia de otros sistemas, éste predice la velocidad apropiada en lugar de la velocidad real, y para ello usamos diferentes sistemas de regresión de inteligencia artificial. El sistema realiza la regresión tomando como entrada una segmentación semántica de la imagen. En concreto, proponemos utilizar el sistema de segmentación conocido como Swiftnet, que combinado con un sistema de regresión empleando la técnica llamada \ac{SPP}, consigue un error medio para la estimación de la velocidad adecuada de tan solo 8.65 km\textbackslash{h}.
%TODO_DONE: Completa esto y luego lo traduces al Inglés abajo.


\vspace{0.5cm}

\textbf{Palabras clave}: visión por computador, inteligencia artificial, sistema de adaptación de la velocidad inteligente, vehículos inteligentes.
%\textbf{Palabras clave}: cinco palabras como máximo, separadas por comas.

%Introduce hoja en blanco
\newpage
\thispagestyle{empty}
\hspace*{0.5cm}
\newpage

\chapter{Abstract}
%\textcolor{red}{Resumen de 100 palabras MÁXIMO - Es OBLIGATORIO}\\
In this work, we show an improved version of the intelligent speed adaptation system  $ISA^{2}$. It consists in the appropriate speed estimation for a vehicle on the basis of the traffic situation at a certain point of time captured by a camera. Unlike other systems, this one predicts the appropriate speed instead of the real speed, that is why we use different artificial intelligence regression systems. The system performs the regression taking as input the semantic segmentation of the image. To be more specific, we propose to use the segmentation system known as Swiftnet, which is combined with a regression system using the \ac{SPP} technique, getting a mean error for the estimation of the appropriate speed of just 8.65 km\textbackslash{h}.

\vspace{0.5cm}

\textbf{Keywords}: computer vision, artificial intelligence, intelligent speed adaptation, intelligent vehicles.
%\textbf{Keywords}: cinco palabras como máximo, separadas por comas.
%Introduce hoja en blanco
\newpage
\thispagestyle{empty}
\hspace*{0.5cm}
\newpage
