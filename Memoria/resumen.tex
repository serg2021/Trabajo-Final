\chapter{Resumen}
%\textcolor{red}{Resumen de 100 palabras MÁXIMO}\\

%En el siguiente trabajo vamos a tratar de solucionar un problema que lleva existiendo desde la invención del automóvil de una forma que no se había visto antes. Con la creación de las normas que regulan el tráfico en las carreteras y la fabricación masiva de vehículos, es un hecho que la probabilidad de que ocurra un accidente es muy alta. Para solucionarlo se inventaron los sistemas \ac{ISA} \cite{reduccion} que ayudaron a reducir este número... Sin embargo, no son perfectos y tienen fallos.

%Es por ello que aquí sugerimos nuestra solución con un nuevo y mejorado sistema: \[ISA^{2}\]
En el siguiente trabajo mostramos una versión mejorada del sistema de adaptación de la velocidad $ISA^{2}$. Éste consiste en la estimación apropiada de la velocidad para un vehículo basándose en la situación del tráfico en un determinado momento. A diferencia de otros sistemas, éste predice la velocidad apropiada en lugar de la velocidad real, y para ello usamos diferentes sistemas de regresión, inteligencia artificial... Utilizaremos además visión por computador para analizar las imágenes referentes al tráfico obtenidas mediante una cámara de fotos.

Todo ello nos servirá como precedente para una futura implementación en vehículos inteligentes.
%TODO_DONE: Sergio, esto es un abstract o resumen, y debes resumir el trabajo. Lo que has puesto no es un resumen. Hay que explicar, muy brevemente, en qué consiste la solucion propuesta. Mira por ejemplo el Abstract/Resumen del artículo de congreso. Cuando lo tengas en castellano lo traduces al inglés y lo subes.

\vspace{0.5cm}

\textbf{Palabras clave}: visión por computador, inteligencia artificial, sistema de adaptación de la velocidad inteligente, vehículos inteligentes.
%\textbf{Palabras clave}: cinco palabras como máximo, separadas por comas.

%Introduce hoja en blanco
\newpage
\thispagestyle{empty}
\hspace*{0.5cm}
\newpage

\chapter{Abstract}
%\textcolor{red}{Resumen de 100 palabras MÁXIMO - Es OBLIGATORIO}\\
In this work, we show an improved version of the intelligent speed adaptation system  $ISA^{2}$. It consists in the appropriate speed estimation for a vehicle on the basis of the traffic situation at a certain point of time. Unlike other systems, this one predicts the appropriate speed instead of the real speed, that is why we use different regression systems, artificial intelligence... In addition, we will use computer vision to analyse the images referred to the traffic situation taken by a camera.

Therefore, all of the above will be used as a precedent for a future implementation in intelligent vehicles.

\vspace{0.5cm}

\textbf{Keywords}: computer vision, artificial intelligence, intelligent speed adaptation, intelligent vehicles.
%\textbf{Keywords}: cinco palabras como máximo, separadas por comas.
%Introduce hoja en blanco
\newpage
\thispagestyle{empty}
\hspace*{0.5cm}
\newpage
