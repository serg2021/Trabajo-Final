\chapter{Resumen}
%\textcolor{red}{Resumen de 100 palabras MÁXIMO}\\

En el siguiente trabajo vamos a tratar de solucionar un problema que lleva existiendo desde la invención del automóvil de una forma que no se había visto antes. Con la creación de las normas que regulan el tráfico en las carreteras y la fabricación masiva de vehículos, es un hecho que la probabilidad de que ocurra un accidente es muy alta. Para solucionarlo se inventaron los sistemas \ac{ISA} \cite{reduccion} que ayudaron a reducir este número... Sin embargo, no son perfectos y tienen fallos.

Es por ello que aquí sugerimos nuestra solución con un nuevo y mejorado sistema: \[ISA^{2}\]
%TODO: Sergio, esto es un abstract o resumen, y debes resumir el trabajo. Lo que has puesto no es un resumen. Hay que explicar, muy brevemente, en qué consiste la solucion propuesta. Mira por ejemplo el Abstract/Resumen del artículo de congreso. Cuando lo tengas en castellano lo traduces al inglés y lo subes.

\vspace{0.5cm}

\textbf{Palabras clave}: visión por computador, inteligencia artificial, sistema de adaptación de la velocidad inteligente, vehículos inteligentes.
%\textbf{Palabras clave}: cinco palabras como máximo, separadas por comas.

%Introduce hoja en blanco
\newpage
\thispagestyle{empty}
\hspace*{0.5cm}
\newpage

\chapter{Abstract}
%\textcolor{red}{Resumen de 100 palabras MÁXIMO - Es OBLIGATORIO}\\



\vspace{0.5cm}

\textbf{Keywords}: computer vision, artificial intelligence, intelligent speed adaptation, intelligent vehicles.
%\textbf{Keywords}: cinco palabras como máximo, separadas por comas.
%Introduce hoja en blanco
\newpage
\thispagestyle{empty}
\hspace*{0.5cm}
\newpage
