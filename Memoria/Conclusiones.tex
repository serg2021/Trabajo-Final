\chapter{Conclusiones}

En este trabajo hemos contribuido a la evolución del sistema $ISA^{2}$ como respuesta a los sistemas \ac{ISA} ya existentes \cite{isa2}. Hemos podido comprobar cómo al implementar un modelo \textbf{Real-Time} con una mayor precisión para la \ac{SS} (Swiftnet \cite{swiftnet}), el propio sistema mejora en aspectos muy importantes, entre los que destacan:

\begin{itemize}
\item Mayor precisión en entornos urbanos para la estimación de la velocidad.
\item Menor tiempo de procesado, lo que posibilita la integración del sistema en un vehículo inteligente.
\end{itemize}

%De modo que, usando este modelo, con las variaciones que ello conlleva, podemos asegurar que este sistema es mejor que su predecesor.
%TODO: Modifica esta frase, no me convence, y pon una comparativa explícita con el anterior modelo en términos de MAE, de modo que el lector termine el documento sabiendo cuanto mejor es la solución propuesta que el anterior. Añade también información sobre la velocidad del sistema en frames por segundo.

Por todo ello, concluimos con la idea de que esta nueva versión de $ISA^{2}$ sirva como punto de referencia para futuras mejoras del mismo. Como futuras líneas de trabajo proponemos las siguientes:
%TODO: Añade, bosqueja, algunas líneas de mejora que se deriven de tu trabajo. Pueden/deben ser muy técnicas.

Al principio de este trabajo, contamos cómo con los sistemas \ac{ISA} se habían reducido cuantiosamente el número de accidentes de tráfico en España y Europa. Con este sistema, y con los que se deriven de éste, esperemos que, en un futuro no muy lejano, ese dato no exista.
