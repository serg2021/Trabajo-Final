\chapter{Conclusiones}

En este trabajo hemos contribuido a la evolución del sistema $ISA^{2}$ como respuesta a los sistemas \ac{ISA} ya existentes \cite{isa2}. Asimismo, hemos podido comprobar cómo al implementar un modelo \textbf{Real-Time} con una mayor precisión para la \ac{SS} (Swiftnet \cite{swiftnet}), el propio sistema mejora en aspectos muy importantes:

\begin{itemize}
\item Mayor precisión en entornos urbanos.
\item Menor coste de implementación.
\item Mayor velocidad de procesado de imágenes para realizar la \ac{SS}.
\end{itemize}

De modo que, usando este modelo, con las variaciones que ello conlleva, podemos asegurar que este sistema es mejor que su predecesor.

Por todo ello, concluimos con la idea de que esta nueva versión de $ISA^{2}$ sirva como punto de referencia para futuras mejoras del mismo. Al principio de este trabajo, contamos cómo con los sistemas \ac{ISA}, se habían reducido cuantiosamente el número de accidentes de tráfico en España... Con este sistema, y con los que se deriven de éste, esperemos que, en un futuro no muy lejano, ese dato no exista.